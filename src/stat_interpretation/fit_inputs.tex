\clearpage
\section{Fit inputs and options}
\label{subsec:fit_options}

For the statistical interpretation, we utilize outputs from the CxAODReader framework, providing histograms of relevant distributions within designated regions to the WSMaker 
\mbox{\texttt{\href{https://gitlab.cern.ch/atlas-physics/sm/ew/vbs-semileptonic-run2/WSMaker\_VBSVV}{WSMaker\_VBSVV}}}
statistical framework for this analysis.

In the fit part of the analysis, the relevant variables are the \mjjtag and the final DNN score distributions.
Table \ref{tab:fitregions_1lep} summarizes the regions and variables used in the fit of the analysis.
These regions are simultaneously fitted using a binned profile likelihood to extract the parameter of interest (POI), which for the SM measurement is the signal strength of the EW VBS production.

The control regions employ bins of \mjjtag with boundaries at \{400, 600, 800, 1000, 1200, 1600, 6000\}~GeV, where the final bin contains all overflow. For the DNN, binning is based on the transformation \(D\), as detailed in reference~\cite{Buscher:2232472}, with a parametrization $Z(signal, background) = (10,5)$.


%%%%%%%%%%%%%%%

%%\begin{table}[h]
%%  \centering
%%  \scalebox{0.8}{
%%    \begin{tabular}{|c|c|c|c|c|c|c|c|c|c|c|c|c|c|c|c|c|c|}
%%      \hline
%%      && \multicolumn{15}{c|}{up-threshold}\\ \hline
%%      Signal region & bin & 1&   2&   3&   4&   5&   6&   7&   8&   9&  10&  11&  12&  13&  14&  15\\ \hline
%%      1L HP      & 0&0.19&0.25&0.30&0.35&0.42&0.49&0.56&\#0.64&0.72&0.79&0.85&0.90&0.94&0.97&1.01\\      
%%      1L LP      & 0&0.13&0.18&0.22&0.26&0.30&0.35&\#0.42&0.49&0.57&0.67&0.76&0.84&0.91&0.96&1.01\\      
%%      1L Tight   & 0&0.09&0.14&0.20&0.27&0.34&0.42&0.50&\#0.58&0.66&0.73&0.80&0.86&0.91&0.95&1.01\\\hline
%%  \end{tabular}
%%  }
%%  \caption{Up thresholds of the bins after applying the binning tranformation D. The marker \# indicates the first bin that needs to be blinded to present a 75\% of data. }
%%  \label{tab:DNNbins}
%%\end{table}

%%%%%%%%%%%%%%%%%%%%%%%%%
%%%Table 
%%%\ref{tab:fitregions_1lep}
%%%reports the summary of the regions and observables used in the fit of the analysis.

\begin{table}[htb!]
  \centering
  \begin{tabular}{lccc}
          \toprule\midrule
          \multirow{2}{*}{Regions} & \multicolumn{3}{c}{1lep channel fit model} \\
          \cmidrule{2-4}
                & Merged high-purity & Merged low-purity & Resolved \\
          \midrule
          SR     & DNN & DNN & DNN \\
          WCR    & \multicolumn{2}{c}{\mjjtag} & \mjjtag \\
          TopCR  & One bin & One bin & \mjjtag \\
          \midrule
          \bottomrule
  \end{tabular}
\caption{\label{tab:fitregions_1lep} Summary of the regions from 1lep channel entering the likelihood of the fit models. 
``One bin'' implies that a single bin without any shape information is used in the corresponding fit region.}
\end{table}

There are two types of nuisance parameters (NPs): those arising from experimental uncertainties and those from MC modeling uncertainties.

For the experimental uncertainties, performance groups have developed various methods to alter particle reconstruction, introducing around a hundred NPs. 
These parameters are subject to Gaussian priors, which reduce the likelihood whenever the fitted values deviate from their nominal values.

For background MC modeling uncertainties, we consider the main background samples like \ttbar, \Wjets, \Zjets, and QCD-produced diboson. We obtain additional NPs by comparing the default generator to an alternative. As shown in Table \ref{tab:NPs:float}, background normalizations are left floating in the fit. The single-top background normalization is constrained with a 30\% Gaussian prior.
For the signal, modeling uncertainties stem from variations in the scale ($\alpha_S$),
PDF, alternative PDF sets, and radiation parameters within the \textsc{Pythia} simulation, as mentioned in section \ref{subsec:sig_uncer}.

\begin{table}[h]
    \centering
\resizebox{0.99\textwidth}{!}{
  \begin{tabular}{|c|c|c| } \hline
    Treatment & NP name & NP name simplified \\ \hline
      Float & ATLAS\_norm\_WjetsMerged                                           &  W norm. Merged \\
      Float & ATLAS\_norm\_WjetsResolved                                         &  W norm. Resolved \\
      Float & ATLAS\_norm\_ZjetsMerged                                           &  Z norm. Merged \\
      Float & ATLAS\_norm\_ZjetsResolved                                         &  Z norm. Resolved \\
      Float & ATLAS\_norm\_ttbarMerged                                           &  $t\bar{t}$ norm. Merged \\
      Float & ATLAS\_norm\_ttbarResolved                                         &  $t\bar{t}$ norm. Resolved \\\hline
      Floating & mu\_SemileptonicVBS                                               &  Signal strength $\mu$(VBS) \\\hline
%%      Floating & mu\_SemileptonicVV                                                &  Signal strength $\mu$(VV) \\\hline

    \end{tabular}
    }
    \caption{Floating parameter (includes signal strength parameter). The normalization factors for \Zjets have minimal impact due to their small contribution in the 1-lepton channel.}
    \label{tab:NPs:float}
\end{table}

