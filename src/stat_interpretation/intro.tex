\clearpage
\section{Introduction}
\label{sec:mll_def}

To determine the signal strength parameter $\mu$ ($\mu_{VBS}$), a binned maximum likelihood fit is used in the statistical analysis. The likelihood function is formulated as follows:
\begin{equation} \label{eq:Lh1}
	\mathcal{L}(N, \tilde{\theta} | \mu, \theta) = \mathrm{Pois}\,(\mu | \mu s + b) \cdot p(\tilde{\theta} | \theta)
\end{equation}
Here, $\mathrm{Pois}\,(\mu | \mu s + b)$ represents the product of Poisson probability terms across all histogram bins:
\begin{equation} \label{eq:Lh2}
	\mathrm{Pois}\,(\mu | \mu s + b) = \prod_{i=1}^{N_{\text{bins}}} \frac{(\mu s_{i}(\theta) + b_{i}(\theta))^{N_{i}} e^{- (\mu s_{i}(\theta) + b_{i}(\theta))}}{N_{i}!}
\end{equation}
In this formulation, $\mu s_{i}$ and $b_{i}$ denote the expected numbers of signal and background events in bin $i$, respectively, while $N_{i}$ represents the number of observed events in that bin. The second part of Equation \ref{eq:Lh1}, $p(\tilde{\theta} | \theta)$, typically referred to as the prior, incorporates our knowledge about the systematic effects considered in the analysis. This binned likelihood approach allows for a robust estimation of $\mu$ while accounting for various statistical and systematic uncertainties.

%%%
%The statistical analysis uses a binned maximum likelihood approach, formulated as a product of Poisson probability terms to evaluate the data. The likelihood function is represented by:
%
%\begin{equation}
%\mathrm{Pois}\,(n|\mu S+B)\left[ \prod_{b\in \text{bins}}^{n} \frac{\mu \nu^{\mathrm{sig}}_{b}+\nu^{\mathrm{bkg}}_{b}}{\mu S+B} \right],
%\end{equation}
%
%In this equation, $\mu$ is the signal strength parameter that scales the expected signal yield $\nu^{\mathrm{sig}}_b$ in each histogram bin $b$. 
%The term $\nu^{\mathrm{bkg}}_b$ indicates the expected background contribution in the same bin.

The sensitivity of the signal and background predictions to systematic uncertainties is quantified through nuisance parameters (NPs), denoted as $\theta$. 
These NPs are typically modeled using either Gaussian or log-normal distributions, with log-normal priors being preferred for normalization uncertainties to ensure that the likelihood remains positive. The expected counts of signal and background events in each bin are modeled as functions of $\theta$. This modeling is structured so that event rates across different categories exhibit a log-normal behavior when $\theta$ is governed by normal distributions.

Priors are used to constrain the NPs towards their nominal values within their assigned uncertainties. This is achieved by incorporating penalty terms or auxiliary measurements into the likelihood function. These additions cause the likelihood to increase whenever an NP deviates significantly from its nominal value. Consequently, the likelihood function $\mathcal{L} (\mu,\theta)$ depends on both the signal strength parameter $\mu$ and the NPs $\theta$. This setup facilitates the integration of our understanding and uncertainties about the system into the analysis, ensuring that the estimation of $\mu$ is consistent with the prior knowledge encapsulated in $\theta$.

The nominal fit result, in terms of the signal strength parameter $\mu$ and its uncertainty $\sigma_{\mu}$, is determined by maximizing the likelihood function across all parameters. This process yields the maximized log-likelihood value (MLL).



