%1-lep DNN ML
\clearpage
\section{Conclusions}

This thesis has presented a search for semileptonic Vector Boson Scattering (VBS) events, utilizing the complete Run-2 data collected by the ATLAS detector at the LHC. A significant focus was placed on my contributions, particularly the deployment of a Deep Neural Network (DNN) approach in analyzing and enhancing the sensitivity to the VBS signal in the \olep channel.


The preliminary unblinded results from this study have shown an observed significance that exceeds the critical 5$\sigma$ threshold, with a reported signal strength parameter $\mu_{VBS} = 1.16 \pm 0.25$.
This achievement is notable, as it suggests the feasibility of observing VBS events through the analysis of the 1-lepton channel alone.


The DNN methodology introduced here serves as an alternative machine learning strategy within the broader context of a combined search across three channels (\zlep, \olep, and \tlep), which utilized a Recurrent Neural Network (RNN) approach. While a direct comparison between the DNN and RNN methods does not definitively establish one as superior to the other, it can be argued based on this work that the simpler, yet effective, DNN approach may offer a more suitable solution for future analyses.

The significance of this work lies in its demonstration of the DNN's potential to enhance the analysis of semileptonic VBS events. By achieving significant results in the 1-lepton channel, this research contributes to the evolving understanding of VBS processes and their observation at the LHC.

This research encountered several limitations, partly due to the extended duration of the combined search across the three channels. This period coincided with the unforeseen challenges posed by the COVID-19 pandemic, which indirectly affected the scope of the investigation. 
This thesis did not account for signal uncertainties arising from EWK-QCD interference and parton shower systematics. More comprehensive studies of the nuisance parameters and alterations in the binning strategies for certain NPs could potentially improve the fit results and overall analysis sensitivity.

The study of anomalous quartic gauge couplings (aQGC) is a significant yet ongoing aspect of the combined search, which is briefly discussed in this thesis. Although my involvement in this specific area was limited, the potential for investigating aQGCs exclusively within the 1-lepton channel, especially with more data from the ATLAS detector, presents an intriguing avenue for future research.

Addressing these limitations presents an opportunity for subsequent research to build upon the foundational work laid out in this thesis, further advancing our understanding and analysis capabilities within the field of semileptonic VBS analysis.




