%%% introduction %%%

The investigation of electroweak symmetry breaking (EWSB) has been one of the pivotal objectives of experiments conducted at the Large Hadron Collider (LHC). 
Within the Standard Model (SM) of particle physics, EWSB is explained by the Brout-Englert-Higgs mechanism, a cornerstone of our understanding of particle interactions~\cite{Englert:1964et,Higgs:1964pj,Guralnik:1964eu}. 
While the Higgs boson's discovery and subsequent measurements have significantly advanced our knowledge, a comprehensive understanding of EWSB requires further investigation.
Vector boson scattering (VBS) plays a crucial role in this context, owing to its sensitivity to the interactions involving the longitudinal components of gauge bosons, which are directly tied to the mechanism of EWSB.

The ATLAS and CMS collaborations have reported results from their fully leptonic VBS analyses conducted during the LHC's Run-II period. These findings, detailed in a series of publications by ATLAS~\cite{STDM-2017-19, STDM-2017-06, STDM-2017-23} and by CMS~\cite{2021135992, 2022137438}, have led to the official claim of observing the SM VBS process with fully leptonic final states.

In the study of VBS, Beyond Standard Model (BSM) physics is often explored through the effective field theory (EFT) framework~\cite{Longhitano:1980tm}. Within this framework, VBS interactions can be influenced by anomalous quartic gauge couplings (aQGCs). Searches for evidence of aQGCs have been conducted across various experiments. These efforts are part of a continuing exploration to understand the potential for new physics beyond what the SM currently describes.
aQGCs typically cause an increase in the VBS cross-section. This enhancement is noticeable at high transverse momentum (\pt) of the vector bosons and at high invariant mass of the two-boson system.

Experimentally, VBS is recognized by two key features: the presence of a pair of vector bosons ($W$, $Z$, or $\gamma$) and two forward jets. These jets are notably separated by a large distance in rapidity and have a high dijet invariant mass. Previous studies searching for aQGCs in VBS events have mainly focused on cases where the bosons decay into leptons ($W(l\nu)$ and $Z(ll)$)~\footnote{Unless otherwise noted, $\ell$ stands for either electron ($e$) or muon ($\mu$) in this thesis.} and photons.

The semi-leptonic channels, such as $V(qq')Z(\nu\nu)$, $V(qq')W(l\nu)$ and $V(qq')Z(ll)$ (V = W, Z), present unique advantages. 
The branching fractions—meaning the probabilities of these decay modes—are much larger for decays involving $V(qq')$ compared to those that are purely leptonic. 
%Furthermore, jet substructure techniques, which analyze the internal structure of jets, enable efficient reconstruction of these events at high \pt, an area highly sensitive to aQGCs. 
This efficiency and sensitivity make the semi-leptonic channels particularly appealing for aQGC searches.

This thesis focuses on analyzing the production of $V(qq')W(l\nu)$ alongside a high-mass dijet system, in a phase space optimized for sensitivity to VBS and aQGCs. This specific setup is referred to as the 1-lepton channel, part of a broader analysis that includes 0-lepton, 1-lepton, and 2-lepton channels~\cite{Charlton:2808769}. The primary emphasis is on the 1-lepton channel, showcasing a machine learning approach designed to improve the detection sensitivity to SM VBS process.

This thesis is structured as follows: Chapter~\ref{ch:theory_overview} offers a brief overview of the Standard Model, delves into the physics of the Higgs field, and explains why searching for VBS and aQGCs is important.
Chapter~\ref{ch:Experimental_Apparatus} describes the experimental setup used to gather the data for this study, focus on the LHC and the ATLAS detector. Chapters~\ref{ch:exp_methods} to~\ref{ch:stat_interpretation} detail the entire workflow of the semi-leptonic VBS analysis and highlight the promising outcomes achieved using my machine learning approach.


