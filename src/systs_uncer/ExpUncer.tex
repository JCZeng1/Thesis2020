%\subsection{Experimental Uncertainties}
\label{subsec:exp_uncer}


\subsection{Baseline uncertainties}
\label{subsubsec:baseline_unc}
The summary of experimental uncertainties is presented in Tab.~\ref{tab:syst_summary_sources_1} to \ref{tab:syst_summary_sources_3}.
 
\begin{table}[!hp]
  \caption{ Qualitative summary of the systematic uncertainties included in this analysis. }
  \label{tab:syst_summary_sources_1}
  \centering
  \footnotesize
  \begin{center}
    \begin{tabular}{|l|l|l|l|}
      \hline
      Source        & Description                     & Analysis Name                         & Notes              \\ \hline
      Electrons     & Energy scale                    &  EG\_SCALE\_ALL                     & \\ 
      Electrons     & Energy resolution               &  EG\_RESOLUTION\_ALL                & \\ 
      Electrons     & Trigger                        &  EL\_EFF\_Trigger\_TOTAL\_1NPCOR\_PLUS\_UNCOR                    & \\ 
      Electrons     & ID efficiency SF                &  EL\_EFF\_ID\_TOTAL\_1NPCOR\_PLUS\_UNCOR  & \\
      Electrons     & Isolation efficiency SF                &   EL\_EFF\_Iso\_TOTAL\_1NPCOR\_PLUS\_UNCOR  & \\
      Electrons     & Reconstruction efficiency SF                &   EL\_EFF\_Reco\_TOTAL\_1NPCOR\_PLUS\_UNCOR  & \\ \hline
      Muons         & \pt\ scale                       &   MUONS\_SCALE                       & \\ 
      Muons         & \pt\ scale (charge dependent)          &   MUON\_SAGITTA\_RHO                    & \\ 
      Muons         & \pt\ scale (charge dependent)          &   MUON\_SAGITTA\_RESBIAS                     & \\ 
      Muons         & \pt\ resolution MS               &   MUONS\_MS                          & \\ 
      Muons         & \pt\ resolution ID               &   MUONS\_ID                          & \\ 
      Muons         & Isolation efficiency SF         &   MUON\_ISO\_SYS                     & \\ 
      Muons         & Isolation efficiency SF         &   MUON\_ISO\_STAT                    & \\ 
      Muons         & Muon reco \& ID efficiency SF               &   MUONS\_EFF\_STAT                          & \\ 
      Muons         & Muon reco \& ID efficiency SF               &   MUONS\_EFF\_STAT\_LOWPT                          & \\ 
      Muons         & Muon reco \& ID efficiency SF               &   MUONS\_EFF\_SYST                          & \\ 
      Muons         & Muon reco \& ID efficiency SF               &   MUONS\_EFF\_SYST\_LOWPT                          & \\ 
      Muons         & Track-to-vertex association efficiency SF         &   MUON\_TTVA\_SYS                     & \\ 
      Muons         & Track-to-vertex association efficiency SF         &   MUON\_TTVA\_STAT                     & \\ 
      Muons         & Trigger         &   MUON\_EFF\_TrigSystUncertainty   & \\
      Muons         & Trigger         &   MUON\_EFF\_TrigStatUncertainty   & \\    \hline
        %MET           & Trigger scale factor            &   METTrigStat                        & \\
        %MET           & Trigger scale factor            &   METTrigTop                         & \\
      MET           & Soft term                       &   MET\_SoftTrk\_ResoPerp             & \\ 
      MET           & Soft term                       &   MET\_SoftTrk\_ResoPara             & \\ 
      MET           & Soft term                       &   MET\_SoftTrk\_Scale              & \\
      %MET           & Jet track uncertainties         &   MET\_JetTrk\_Scale                 & \\ 
      \hline
      \end{tabular}
    \end{center}
  \end{table}

%\textcolor{red}{cross-check with latest fit inputs}

\begin{table}[!hp]
  \caption{ Qualitative summary of the systematic uncertainties included in this analysis. }
  \label{tab:syst_summary_sources_2}
  \centering
  \footnotesize
  \begin{center}
    \begin{tabular}{|l|l|l|l|}
      \hline
      Source        & Description                     & Analysis Name                         & Notes              \\ \hline
      Small-R Jets  & JES category reduction            &  JET\_CR\_JET\_BJES\_Response                             & \\ 
      Small-R Jets  & JES category reduction            &  JET\_CR\_JET\_EffectiveNP\_Detector1                     & \\ 
      Small-R Jets  & JES category reduction            &  JET\_CR\_JET\_EffectiveNP\_Detector2                     & \\ 
      Small-R Jets  & JES category reduction            &  JET\_CR\_JET\_EffectiveNP\_Mixed1                        & \\ 
      Small-R Jets  & JES category reduction            &  JET\_CR\_JET\_EffectiveNP\_Mixed2                        & \\ 
      Small-R Jets  & JES category reduction            &  JET\_CR\_JET\_EffectiveNP\_Mixed3                        & \\ 
      Small-R Jets  & JES category reduction            &  JET\_CR\_JET\_EffectiveNP\_Modelling1                    & \\ 
      Small-R Jets  & JES category reduction            &  JET\_CR\_JET\_EffectiveNP\_Modelling2                    & \\ 
      Small-R Jets  & JES category reduction            &  JET\_CR\_JET\_EffectiveNP\_Modelling3                    & \\ 
      Small-R Jets  & JES category reduction            &  JET\_CR\_JET\_EffectiveNP\_Modelling4                    & \\ 
      Small-R Jets  & JES category reduction            &  JET\_CR\_JET\_EffectiveNP\_Statistical1                  & \\ 
      Small-R Jets  & JES category reduction            &  JET\_CR\_JET\_EffectiveNP\_Statistical2                  & \\ 
      Small-R Jets  & JES category reduction            &  JET\_CR\_JET\_EffectiveNP\_Statistical3                  & \\ 
      Small-R Jets  & JES category reduction            &  JET\_CR\_JET\_EffectiveNP\_Statistical4                  & \\ 
      Small-R Jets  & JES category reduction            &  JET\_CR\_JET\_EffectiveNP\_Statistical5                  & \\ 
      Small-R Jets  & JES category reduction            &  JET\_CR\_JET\_EffectiveNP\_Statistical6                  & \\ 
      Small-R Jets  & JES category reduction            &  JET\_CR\_JET\_Flavor\_Composition                        & \\ 
      Small-R Jets  & JES category reduction            &  JET\_CR\_JET\_Flavor\_Response                           & \\ 
      Small-R Jets  & JES category reduction            &  JET\_CR\_JET\_Pileup\_OffsetMu                           & \\ 
      Small-R Jets  & JES category reduction            &  JET\_CR\_JET\_Pileup\_OffsetNPV                          & \\ 
      Small-R Jets  & JES category reduction            &  JET\_CR\_JET\_Pileup\_PtTerm                             & \\ 
      Small-R Jets  & JES category reduction            &  JET\_CR\_JET\_Pileup\_RhoTopology                        & \\ 
      Small-R Jets  & JES category reduction            &  JET\_CR\_JET\_PunchThrough\_MC16                         & \\ 
      Small-R Jets  & JES category reduction            &  JET\_CR\_JET\_SingleParticle\_HighPt                     & \\ 
      Small-R Jets  & JES category reduction            &  JET\_CR\_JET\_EtaIntercalibration\_TotalStat             & \\ 
      Small-R Jets  & JES category reduction            &  JET\_CR\_JET\_EtaIntercalibration\_Modelling             & \\ 
      Small-R Jets  & JES category reduction            &  JET\_CR\_JET\_EtaIntercalibration\_NonClosure\_highE     & \\ 
      Small-R Jets  & JES category reduction            &  JET\_CR\_JET\_EtaIntercalibration\_NonClosure\_negEta    & \\ 
      Small-R Jets  & JES category reduction            &  JET\_CR\_JET\_EtaIntercalibration\_NonClosure\_posEta    & \\ 
        \hline
        Small-R Jets  & JER                  &  JET\_CR\_JET\_JER\_DataVsMC                  & \\ 
        Small-R Jets  & JER                  &  JET\_CR\_JET\_JER\_EffectiveNP\_1            & \\ 
        Small-R Jets  & JER                  &  JET\_CR\_JET\_JER\_EffectiveNP\_2            & \\ 
        Small-R Jets  & JER                  &  JET\_CR\_JET\_JER\_EffectiveNP\_3            & \\ 
        Small-R Jets  & JER                  &  JET\_CR\_JET\_JER\_EffectiveNP\_4            & \\ 
        Small-R Jets  & JER                  &  JET\_CR\_JET\_JER\_EffectiveNP\_5            & \\ 
        Small-R Jets  & JER                  &  JET\_CR\_JET\_JER\_EffectiveNP\_6            & \\ 
        Small-R Jets  & JER                  &  JET\_CR\_JET\_JER\_EffectiveNP\_7restTerm    & \\ 
        \hline
        Small-R Jets  & JVT                  &  JET\_JvtEfficiency    & \\
%        Small-R Jets  & fJVT                  &  JET\_fJvtEfficiency    & \\ 
        \hline
        \end{tabular}
    \end{center}
  \end{table}

%\textcolor{red}{cross-check with latest fit inputs}

\begin{table}[!hp]
  \caption{ Qualitative summary of the systematic uncertainties included in this analysis. }
  \label{tab:syst_summary_sources_3}
  \centering
  \footnotesize
  \begin{center}
    \begin{tabular}{|l|l|l|l|}
      \hline
      Source        & Description                     & Analysis Name                         & Notes              \\ \hline
      Large-R Jets  & \pt scale                       & FATJET\_Medium\_JET\_Rtrk\_Baseline\_pT                            &  \\
      %Large-R Jets  & \pt scale                       & FATJET\_Medium\_JET\_Rtrk\_Closure\_pT                             &  \\
      Large-R Jets  & \pt scale                       & FATJET\_Medium\_JET\_Rtrk\_Modelling\_pT                           &  \\
      Large-R Jets  & \pt scale                       & FATJET\_Medium\_JET\_Rtrk\_TotalStat\_pT                           &  \\
      Large-R Jets  & \pt scale                       & FATJET\_Medium\_JET\_Rtrk\_Tracking\_pT                            &  \\

      Large-R Jets  & \pt scale                       & FATJET\_BJT\_JET\_EtaIntercalibration\_Modelling      & \\
      Large-R Jets  & \pt scale                       & FATJET\_BJT\_JET\_Flavor\_Composition                 & \\
      Large-R Jets  & \pt scale                       & FATJET\_BJT\_JET\_Flavor\_Response                    & \\

      Large-R Jets  & Mass resolution                 & FATJET\_JMR                            & \\\hline
      Large-R Jets  & JER                             & FATJET\_JER                            & \\\hline

%
      B-tagging     & Flavor tagging scale factors    &  FT\_EFF\_Eigen\_B\_0\_AntiKt4PFlowJets                                & \\
      B-tagging     & Flavor tagging scale factors    &  FT\_EFF\_Eigen\_B\_1\_AntiKt4PFlowJets                                & \\
      B-tagging     & Flavor tagging scale factors    &  FT\_EFF\_Eigen\_B\_2\_AntiKt4PFlowJets                                & \\
      B-tagging     & Flavor tagging scale factors    &  FT\_EFF\_Eigen\_C\_0\_AntiKt4PFlowJets                                & \\
      B-tagging     & Flavor tagging scale factors    &  FT\_EFF\_Eigen\_C\_1\_AntiKt4PFlowJets                                & \\
      B-tagging     & Flavor tagging scale factors    &  FT\_EFF\_Eigen\_C\_2\_AntiKt4PFlowJets                                & \\
      B-tagging     & Flavor tagging scale factors    &  FT\_EFF\_Eigen\_C\_3\_AntiKt4PFlowJets                                & \\
      B-tagging     & Flavor tagging scale factors    &  FT\_EFF\_Eigen\_Light\_0\_AntiKt4PFlowJets                            & \\
      B-tagging     & Flavor tagging scale factors    &  FT\_EFF\_Eigen\_Light\_1\_AntiKt4PFlowJets                            & \\
      B-tagging     & Flavor tagging scale factors    &  FT\_EFF\_Eigen\_Light\_2\_AntiKt4PFlowJets                            & \\
      B-tagging     & Flavor tagging scale factors    &  FT\_EFF\_Eigen\_Light\_3\_AntiKt4PFlowJets                            & \\
      B-tagging     & Flavor tagging scale factors    &  FT\_EFF\_extrapolation\_AntiKt4PFlowJets                              & \\
      B-tagging     & Flavor tagging scale factors    &  FT\_EFF\_extrapolation\_from\_charm\_AntiKt4PFlowJets                 & \\

      \hline                          
      Pileup reweighting & PRW\_DATASF & PRW\_DATASF &\\ 
      Luminosity & LumiNP & ATLAS\_LUMI\_2015\_2018 & \\
\hline
\end{tabular}
    \end{center}
  \end{table}

\clearpage
\subsubsection*{Luminosity}
%%The uncertainty on the integrated luminosity for the 2015+2016 dataset is 2.1\%,
%%and 2.4\% for the 2017 dataset.
%%The uncertainty for the 2018 data alone is 2.0\%, and the uncertainty for the combined run-2 dataset (2015-2018) is 1.7\% \cite{AtlasLumiRun2}.
The luminosity uncertainty is applied to those backgrounds estimated from simulation and the signal samples.
The uncertainties on the integrated luminosity for the datasets are as follows:
\begin{itemize}
    \item 2015+2016 dataset: 2.1\%
    \item 2017 dataset: 2.4\%
    \item 2018 dataset: 2.0\%
    \item Combined Run-2 dataset (2015-2018): 1.7\%
\end{itemize}
Reference: \cite{AtlasLumiRun2}.


\subsubsection*{Pileup reweighting}
The uncertainty associated with the pileup reweighting is accounted for as \texttt{PRW\_DATASF} \cite{ExtendedPileupReweighting}. 
Additionally, a variation in the pileup reweighting of MC simulations is included to address the uncertainty in the ratio of the predicted to the measured inelastic cross-section within the fiducial volume defined by $M_X > 13\,\GeV$, where $M_X$ is the mass of the non-diffractive hadronic system \cite{STDM-2015-05}.

%%The uncertainty associated with the pileup reweighting is considered\cite{ExtendedPileupReweighting} as PRW\_DATASF.
%%A variation in the pileup reweighting of MC is included to cover the uncertainty on the ratio between the predicted
%%and measured inelastic cross-section in the fiducial volume defined by $M_X > 13\,\GeV$ where $M_X$ is the mass
%%of the non-diffractive hadronic system~\cite{STDM-2015-05}.


\subsubsection*{Trigger}
Systematic uncertainties on the efficiency of the electron or muon triggers are evaluated using the tag and probe method, applied to backgrounds estimated from simulation and signal samples. 
The efficiencies are obtained using the \texttt{ElectronEfficiencyCorrection} \cite{AsgElectronEfficiencyCorrectionTool} and \texttt{MuonEfficiencyCorrections} \cite{TrigMuonEfficiency}. The uncertainty in the \met trigger is derived from the scale factor estimation, incorporating statistical contributions and efficiency discrepancies between MC samples, specifically ttbar and $W$+jets \cite{ATLAS-CONF-2016-091, Masubuchi:2151844}. Upon evaluation, trigger uncertainties were found to be less than 1\% and were subsequently excluded from the final fit results.

%%Systematic uncertainties on the efficiency of the electron or muon triggers are evaluated using
%%the tag and probe method. It is applied to those backgrounds estimated from simulation and the signal samples.
%%We use \texttt{ElectronEfficiencyCorrection}~\cite{AsgElectronEfficiencyCorrectionTool} and
%%\texttt{MuonEfficiencyCorrections}~\cite{TrigMuonEfficiency} to obtain them. 
%%The uncertainty from \met trigger arises from the estimation on scale factor which contains two contributions:
%%statistics and the efficiency discrepancy between MC samples (ttbar and $W$+jets) \cite{ATLAS-CONF-2016-091, Masubuchi:2151844}.
%%After trigger uncertainties were evaluated, the effects were found to be less than 1\%. Therefore they were pruned in the final fit result.

\subsubsection*{Muons and electrons}
The following systematic uncertainties are applied to electrons and muons for simulation-based estimations:

\begin{itemize}
    \item \textbf{Identification and reconstruction efficiencies:} Measured using the tag and probe method centered around the $Z$ mass peak.
    \item \textbf{Isolation efficiency:} The scale factor and its uncertainty are derived via the tag and probe method, also utilizing the $Z$ mass peak.
    \item \textbf{Energy and Momentum scales:} Determined through the $Z$ mass line shape analysis, with contributions from the CP groups.
    \item \textbf{Track-to-vertex association efficiency:} This applies solely to muons.
\end{itemize}

Implementation of these uncertainties is carried out through the following tools:
\begin{itemize}
    \item \texttt{ElectronPhotonFourMomentumCorrection} \cite{EgammaCalibration}
    \item \texttt{ElectronEfficiencyCorrection} \cite{AsgElectronEfficiencyCorrectionTool}
%%    \item \texttt{MuonMomentumCorrections}
    \item \texttt{MuonMomentumCorrections} and \texttt{MuonEfficiencyCorrections} \cite{MCPAnalysisGuidelines}
\end{itemize}

%%The following systematic uncertainties are applied to electrons and muons in estimations based on the simulation:
%%
%%\begin{itemize}
%%\item Identification and reconstruction efficiencies: The efficiencies are measured with the tag and probe method using the $Z$ mass peak.
%%\item Isolation efficiency: Scale factor and its uncertainty are derived by tag and probe method using the $Z$ mass peak as well.
%%\item Energy and Momentum scales: These are also measured with $Z$ mass line shape, and provided by the CP groups.
%%\item Track-to-vertex association efficiency: Only for muons.
%%\end{itemize}
%%
%%They are implemented in \texttt{ElectronPhotonFourMomentumCorrection}~\cite{EgammaCalibration},\\
%%\texttt{ElectronEfficiencyCorrection}~\cite{AsgElectronEfficiencyCorrectionTool}, \\
%%\texttt{MuonMomentumCorrections} and \texttt{MuonEfficiencyCorrections}~\cite{MCPAnalysisGuidelines}.


\subsubsection*{Missing transverse energy}
The missing transverse energy (\met) is calculated using physics objects, as outlined in Section~\ref{sec:MET_reconstruction}. 
Consequently, systematic uncertainties in the reconstructed components, such as the jet energy scale, directly affect the \met, representing the primary sources of its uncertainty. 
Additionally, the uncertainty referred to as the ``Soft Term'' arises from tracks in the inner detector that are not associated with any reconstructed object.
The resolution and scale of the Soft Term are varied within their respective uncertainties to assess their impact on the total \met uncertainty, utilizing the \texttt{METUtilities} tool \cite{METUtilSystematics}.


% The missing transverse energy is calculated using physics objects as described in Section~\ref{sec:ObjectDefinition}. 
%from the negative vectorial sum of physics objects: muons, electrons, taus, photons, jets and unassociated clusters of calorimeter cells. 
%%As such, all of the systematic errors on the reconstructed components, e.g. the jet energy scale,
%%result in an uncertainty on \met. These are the dominant sources of uncertainty on \met.
%%In addition, there is an uncertainty called the ''Soft Term'', from the unassociated tracks.
%%The resolution and scale of this soft term are varied within their errors to evaluate their
%%contribution to the total uncertainty using \texttt{METUtilities}~\cite{METUtilSystematics}.

%\subsubsection*{ Track missing transverse energy}
%A very loose cut on the track missing \et\ is applied for the event cleaning in 0-lep channel.
%The impact of possible mis-modelling of track-\met\ on the total background yield is estimated by varying the track-\met value by $\pm 2\%$ and found to be negligible.
%Details can be found at \url{https://indico.cern.ch/event/850077/contributions/3575954/attachments/1913191/3178067/track-met-study-2.pdf}.

\subsubsection*{Small-$R$ Jet Energy Scale and Resolution Uncertainty}
The jet energy scale (JES) and resolution (JER) for small-R jets are determined in situ by comparing the response in MC to data across various bins in kinematic phase space, employing the \texttt{JetUncertainties} tool~\cite{JetUncertainties}. 
The analysis utilizes the configuration \texttt{R4\_CategoryReduction\_SimpleJER.config}, incorporating approximately 30 JES and 8 JER uncertainty components. These uncertainties are also significant in the boosted analysis due to their impact on the calculation of the \met. Additionally, the Jet Vertex Tagger (JVT) efficiency uncertainty is evaluated using \texttt{JetJvtEfficiency} \cite{JVTCalib}; however, its effect was found to be below 1\%, leading to its exclusion from the final fit.


%The jet energy scale and resolution of the small-R jets are measured in situ by calculating
%the response between MC and data in various bins of kinematic phase space using \texttt{JetUncertainties}~\cite{JetUncertainties}. 
%We use the configuration \texttt{R4\_CategoryReduction\_SimpleJER.config}, 
%with roughly 30 JES uncertainty components and 8 JER uncertainty components.
%They also enter the boosted analysis because they are used in the calculation of the missing transverse energy.
%We also considered the uncertainty on JVT efficiency, using \texttt{JetJvtEfficiency}~\cite{JVTCalib}, 
%but the effects were under 1\%, so the uncertainty was pruned in the final fit.


\subsubsection*{Large-$R$ Jet Energy Scale and Resolution Uncertainty}
\label{sec:fatjetUncert}
The uncertainties for the large-\(R\) jet energy scale are incorporated according to the prescription provided by the jet substructure group, as included in the \texttt{JetUncertainties} package~\cite{JSSrecommendation}. 
The uncertainty related to the jet \pt scale is assessed through the Rtrk method, which involves a comparison of the jet \pt to track-jet \pt ratio in dijet data versus simulation. 
Beyond the ``baseline'' uncertainty, additional considerations include uncertainties related to track measurements (``Tracking''), variations between Pythia and Sherpa dijet simulations (``Modelling''), and the statistical uncertainty in dijet data (``TotalStat'').
More information are provided in the main twiki for the large-R jets recommendations \cite{JSSrecommendation2}.

%%The large-$R$ jet energy scale uncertainties are included following the prescription of the
%%jet substructure group included in the \texttt{JetUncertainties} package~\cite{JSSrecommendation}. 
%%The uncertainty on the \pt scale of jets is evaluated by
%%comparing the ratio of the jet \pt to track-jet \pt in dijet data and simulation (Rtrk method).
%%In addition to this ``Baseline'' uncertainty, the uncertainties on track measurements (``Tracking''), 
%%differences between Pythia and Sherpa dijet simulations (``Modelling'') 
%%and the statistical uncertainty of dijet data (``TotalStat'') are considered.
%%More information are provided in the main twiki for the large-R jets recommendations \cite{JSSrecommendation2}.

%The large-$R$ jet resolution uncertainty recommendation is not included in the \texttt{JetUncertainties} package~\cite{JSSrecommendation}.
%As a \pt resolution uncertainty, jet \pt is smeared by Gaussian with 2\% width.


%\subsubsection{$W/Z$-tagging efficiency SF Uncertainty}
%On the other hand, we cannot use Rtrk method to evaluate mass and \DTwoBetaOne scale uncertainties, because TCC algorithm uses track measurements to reconstruct jet substructure variables.
%In order to avoid the possible bias, therefore, the efficiency of $W/Z$-tagging based on cuts on jet mass and \DTwoBetaOne is estimated in data using the control sample and corrected by comparing it with simulation.
%The efficiency to $W/Z$-induced jet signal is estimated by \ttbar control sample, while the efficiency to single-$q/g$ background is estimated by dijet sample.
%The effects of experimental and theoretical uncertainties on the efficiency SF is studied.
%By taking the double ratio (ratio of efficiencies between data and simulation), the uncertainties not coming from jet mass and \DTwoBetaOne scale/resolution are cancel out.
%The efficiency SF and uncertainties on it are estimated in each of (1) pass mass and pass \DTwoBetaOne (HP SR), (2) pass mass and fail \DTwoBetaOne (LP SR), (3) fail mass and pass \DTwoBetaOne (HP CR) and (4) fail mass and fail \DTwoBetaOne (LP CR) regions, which are used to define SR and CR in our analysis, and correlation between four regions are correctly taken into account.

\subsubsection*{B-tagging systematics}
Systematic uncertainties related to $b$-tagging are accounted for as described in \cite{BTagCalib}. 
These uncertainties stem from scaling factors that adjust for any discrepancies in $b$-tagging efficiency between data and MC. 
Separated scale factors and their respective systematic uncertainties are determined for jets originating from $b$-quarks, $c$-quarks, and light flavor quarks, drawing on various measurements.

%%The systematic uncertainties associated to the $b$-tagging are considered\cite{BTagCalib}.
%%They are evaluated as uncertainties on the scaling factor to take account for possible disagreement of the $b$-tag efficiency between data and MC.
%%Separated scale factors and corresponding systematic uncertainties are provided for $b$-, $c$- and light-flavor-induced jets, based on several measurements.
 
%%%
\subsection{$W/Z$-tagging efficiency SF Uncertainty}
%\subsubsection{Boson Tagger}
\label{subsec:bkg_uncer_vtagger}
For the uncertainties associated with the boson tagger's background efficiency, we consider both the large-$R$ jet-related uncertainties and the modeling uncertainties from multijet and $\gamma$+jets processes. The modeling uncertainties are estimated by comparing the nominal Pythia 8 and Sherpa samples for multijet and $\gamma$+jets, respectively, against their alternative samples. For more details on how the tagger is defined, see Section~\ref{subsubsec:merged_jets_selection}.
%\cite{ATL-PHYS-PUB-2020-017}.

Systematic uncertainties associated with the scale factors, which assess the boson tagger’s relative performance in data versus MC, are thoroughly evaluated. These uncertainties include various aspects: for background, considerations include matrix element variations, hadronization, radiation effects, and the impacts from dijets or $\gamma$+jets events. For the signal, uncertainties include considerations like extrapolation at high \pt.

Plots illustrating the impact of uncertainties on the Data/MC ratio for the \olep channel are presented in Figure \ref{fig:1LepVTaggerUnc}.

\begin{figure}[ht]
    \centering
    \begin{subfigure}[b]{0.32\textwidth}
        \centering
        \includegraphics[width=\textwidth]{figures/1lep/VTaggerUnc/VTagCRTopHPtagMjj_SystBreakDown.pdf}
        \caption{Merged HP TopCR}
        \label{fig:MergedHPTopCR}
    \end{subfigure}
    \begin{subfigure}[b]{0.32\textwidth}
        \centering
        \includegraphics[width=\textwidth]{figures/1lep/VTaggerUnc/VTagCRTopLPtagMjj_SystBreakDown.pdf}
        \caption{Merged LP TopCR}
        \label{fig:MergedLPTopCR}
    \end{subfigure}
    \begin{subfigure}[b]{0.32\textwidth}
        \centering
        \includegraphics[width=\textwidth]{figures/1lep/VTaggerUnc/VTagCRVjetMergedtagMjj_SystBreakDown.pdf}
        \caption{Merged Wjets CR}
        \label{fig:MergedWjetsCR}
    \end{subfigure}
    \caption{Comparison of boson tagging scale factors in the 1-lepton channel.}
    \label{fig:1LepVTaggerUnc}
\end{figure}




%%%
\subsection{Quark/Gluon jets uncertainty}
%\subsubsection{Quark/Gluon jets uncertainty}
\label{subsec:bkg_uncer_qg}

Jet flavor response and composition uncertainties consider the different responses of quark- and gluon-initiated jets. 
The response uncertainties are centrally derived from dijet events, using alternative MC samples, specifically Pythia 8 and Herwig++. 
For flavor composition, uncertainties are typically assumed based on a presumed 50/50 quark/gluon mix, with a conservative approach that assumes 100\% uncertainty. However, in VBS topologies, which tend to be quark-enriched, these uncertainties can limit measurement sensitivity. 
To mitigate this, we analyze the gluon fraction within our analysis phase-space. This analysis allows us to rederive jet flavor-related uncertainties using custom gluon fractions, potentially reducing these uncertainties.

The estimation of the gluon fraction across various analysis regions and samples is performed as a function of the small-$R$ jet $\pt$ and $\eta$. This is achieved by utilizing truth parton label information in MC samples to discern the proportions of quarks and gluons. For the purpose of quark estimation, all jets except those initiated by b-quarks are considered in the denominator.

Two-dimensional histograms, representing the gluon fraction as a function of jet $\pt$ and $\eta$, serve as inputs for recalculating jet flavor and composition uncertainties. Each MC sample is associated with a single input file. The gluon fraction for a specific ($\pt$, $\eta$) bin is determined by aggregating data across all regions.

To account for the uncertainty in the gluon fraction, additional inputs are utilized. The uncertainty for a particular ($\pt$, $\eta$) bin, $\sigma_{gfrac}$, is calculated as follows:
\[
\sigma_{gfrac} = \sqrt{\sigma_{region}^{2} + \sigma_{gen}^{2}}
\]
Here, \(\sigma_{region}\) represents the maximal deviation in gluon fraction between the nominal and any analysis region. Meanwhile, \(\sigma_{gen}\) denotes the generator uncertainty, obtained from comparing alternative Pythia 8 and Herwig++ MC samples.

The inputs for the gluon fraction and their corresponding uncertainties in the \olep channel are illustrated in Figures \ref{fig:QGFracInputs2D1Lep} and \ref{fig:QGFracErrorInputs2D1Lep}, respectively. 
Comparison between old flavor uncertainties and newly derived flavor uncertainties is shown is Figure \ref{fig:1LepFlavorVarOldNew}. It is evident from this comparison that flavor composition uncertainties, in general, are reduced.

%%%%%%%%%%%%%


\begin{figure}[p]
    \centering
    \begin{subfigure}[b]{0.3\textwidth}
        \centering
        \includegraphics[width=\textwidth]{figures/QGfrac/GluonFrac2D_1LepSignal.png}
        \caption{Signal samples}
        \label{fig:GluonFracSignal}
    \end{subfigure}
    \hfill
    \begin{subfigure}[b]{0.3\textwidth}
        \centering
        \includegraphics[width=\textwidth]{figures/QGfrac/GluonFrac2D_1LepWjets.png}
        \caption{\Wjets samples}
        \label{fig:GluonFracWjets}
    \end{subfigure}
    \hfill
    \begin{subfigure}[b]{0.3\textwidth}
        \centering
        \includegraphics[width=\textwidth]{figures/QGfrac/GluonFrac2D_1Lepttbar.png}
        \caption{ttbar samples}
        \label{fig:GluonFracttbar}
    \end{subfigure}

    \bigskip

    \begin{subfigure}[b]{0.3\textwidth}
        \centering
        \includegraphics[width=\textwidth]{figures/QGfrac/GluonFrac2D_1LepDiboson.png}
        \caption{Diboson samples}
        \label{fig:GluonFracDiboson}
    \end{subfigure}
%    \hfill
    \begin{subfigure}[b]{0.3\textwidth}
        \centering
        \includegraphics[width=\textwidth]{figures/QGfrac/GluonFrac2D_1LepSingleTop.png}
        \caption{Single top samples}
        \label{fig:GluonFracSingleTop}
    \end{subfigure}
    \hfill

    \caption{Gluon fraction inputs used to calculate jet uncertainties for the 1 lepton channel.}
    \label{fig:QGFracInputs2D1Lep}
\end{figure}


\begin{figure}[p]
    \centering
    \begin{subfigure}[b]{0.3\textwidth}
        \centering
        \includegraphics[width=\textwidth]{figures/QGfrac/GluonFracError2D_1LepSignal.png}
        \caption{Signal samples}
        \label{fig:GluonFracErrorSignal}
    \end{subfigure}
    \hfill
    \begin{subfigure}[b]{0.3\textwidth}
        \centering
        \includegraphics[width=\textwidth]{figures/QGfrac/GluonFracError2D_1LepWjets.png}
        \caption{\Wjets samples}
        \label{fig:GluonFracErrorWjets}
    \end{subfigure}
    \hfill
    \begin{subfigure}[b]{0.3\textwidth}
        \centering
        \includegraphics[width=\textwidth]{figures/QGfrac/GluonFracError2D_1Lepttbar.png}
        \caption{ttbar samples}
        \label{fig:GluonFracErrorttbar}
    \end{subfigure}

    \bigskip

    \begin{subfigure}[b]{0.3\textwidth}
        \centering
        \includegraphics[width=\textwidth]{figures/QGfrac/GluonFracError2D_1LepDiboson.png}
        \caption{Diboson samples}
        \label{fig:GluonFracErrorDiboson}
    \end{subfigure}
%    \hfill
    \begin{subfigure}[b]{0.3\textwidth}
        \centering
        \includegraphics[width=\textwidth]{figures/QGfrac/GluonFracError2D_1LepSingleTop.png}
        \caption{Single top samples}
        \label{fig:GluonFracErrorSingleTop}
    \end{subfigure}
    \hfill

    \caption{Gluon fraction error inputs used to calculate jet uncertainties for the 1 lepton channel.}
    \label{fig:QGFracErrorInputs2D1Lep}
\end{figure}


\begin{figure}[ht]
    \centering
    \begin{subfigure}[b]{0.4\textwidth}
        \centering
        \includegraphics[width=\textwidth]{figures/1lep/FlavorVar/SystFCompCRTopResTight_All_tagMjj.png}
        \caption{Resolved TopCR Flavour Composition}
        \label{fig:ResolvedTopCRFlavourComposition}
    \end{subfigure}
    \quad % This adds some spacing between the figures in the same row
    \begin{subfigure}[b]{0.4\textwidth}
        \centering
        \includegraphics[width=\textwidth]{figures/1lep/FlavorVar/SystFResCRTopResTight_All_tagMjj.png}
        \caption{Resolved TopCR Flavour Response}
        \label{fig:ResolvedTopCRFlavourResponse}
    \end{subfigure}

    \bigskip % This adds extra vertical spacing between the rows of figures

    \begin{subfigure}[b]{0.4\textwidth}
        \centering
        \includegraphics[width=\textwidth]{figures/1lep/FlavorVar/SystFCompCRVjetsResTight_All_tagMjj.png}
        \caption{Resolved WjetsCR Flavour Composition}
        \label{fig:ResolvedWjetsCRFlavourComposition}
    \end{subfigure}
    \quad % This adds some spacing between the figures in the same row
    \begin{subfigure}[b]{0.4\textwidth}
        \centering
        \includegraphics[width=\textwidth]{figures/1lep/FlavorVar/SystFResCRVjetsResTight_All_tagMjj.png}
        \caption{Resolved WjetsCR Flavour Response}
        \label{fig:ResolvedWjetsCRFlavourResponse}
    \end{subfigure}

    \caption{Comparison between old flavor uncertainties and newly derived flavor uncertainties.}
    \label{fig:1LepFlavorVarOldNew}
\end{figure}


%%%
\subsection{Tracks uncertainties}
\label{subsec:tracks_uncer}

We incorporate dedicated uncertainties related to track reconstruction, particularly for tracks associated with small-R jets, as outlined in the recommendations on \mbox{\texttt{\href{https://twiki.cern.ch/twiki/bin/viewauth/AtlasProtected/QuarkGluonTagging\#Current_Recommendations}{QuarkGluonTagging}}} and \mbox{\texttt{\href{https://twiki.cern.ch/twiki/bin/view/AtlasProtected/TrackingCPRecsRun2Final\#Track_Systematics_Tools}{TrackingCPRecsRun2Final}}}. 
%Specifically, the ghost-associated track multiplicities of our jets are utilized as inputs to the recurrent layer of the final network.

The procedure in \cite{ATL-PHYS-PUB-2017-009} is used to derive these track multiplicity related uncertainties, of which there are five main components:
\begin{itemize}
    \item fake track efficiency: the uncertainty from the track mis-reconstruction rate;
    \item track efficiency: the uncertainty from the track reconstruction efficiency;
    \item experimental: the uncertainty in charged particle multiplicity, derived as in \cite{CERN-PH-EP-2016-001};
    \item PDF: the theory uncertainty from the parton distribution function;
    \item matrix element: the theory uncertainty from the matrix element calculation;
\end{itemize}
The impacts from these uncertainties are expected to be low.

%%%%%%%%%%%%%%%
%%%We take into account some dedicated uncertainties on the tracks' reconstruction, 
%%%in particular, for the tracks associated to the small-R jets 
%%%\mbox{\texttt{\href{https://twiki.cern.ch/twiki/bin/viewauth/AtlasProtected/QuarkGluonTagging\#Current_Recommendations}{QuarkGluonTagging}}}
%%%\mbox{\texttt{\href{https://twiki.cern.ch/twiki/bin/view/AtlasProtected/TrackingCPRecsRun2Final\#Track_Systematics_Tools}{TrackingCPRecsRun2Final}}}
%%%. 
%%%Indeed, we use the track multiplicities of the ghost associated tracks to our jets as input to the recurrent layer of the final network.
%%%
%%%The procedure in ~\cite{ATL-PHYS-PUB-2017-009}
%%%is used to derive these track multiplicity related uncertainties,
%%%of which there are five main components:
%%%\begin{itemize}
%%%    \item fake track efficiency: uncertainty from track mis-reconstruction rate;
%%%    \item track efficiency: uncertainty from track reconstruction efficiency;
%%%    \item experimental: uncertainty in charged particle multiplicity, derived as in ~\cite{CERN-PH-EP-2016-001};
%%%    \item PDF: theory uncertainty from parton distribution function;
%%%    \item matrix element: theory uncertainty from matrix element calculation;
%%%\end{itemize}
%%%Figures~\ref{fig:1lep_TrackUncCR}, \ref{fig:1lep_TrackUncSR} shows the variations from these sources of
%%%track multiplicity uncertainties for background and signal events in the \olep signal regions.
%%%The impacts from these uncertainties are expected to be low, as they are less than 5\% in the high RNN
%%%bins.

%%%\begin{figure}[ht]
%%%    \centering
%%%        \subfigure[Merged HP SR \ttbar]{\includegraphics[width=0.45\textwidth]{figures/1lep/TrackSyst/SystQGSRHP_ttbar_RNN.pdf}}
%%%        \subfigure[Merged HP SR W]{\includegraphics[width=0.45\textwidth]{figures/1lep/TrackSyst/SystQGSRHP_W_RNN.pdf}}
%%%        \subfigure[Merged LP SR \ttbar]{\includegraphics[width=0.45\textwidth]{figures/1lep/TrackSyst/SystQGSRLP_ttbar_RNN.pdf}}
%%%        \subfigure[Merged LP SR W]{\includegraphics[width=0.45\textwidth]{figures/1lep/TrackSyst/SystQGSRLP_W_RNN.pdf}}
%%%        \subfigure[Resolved SR \ttbar]{\includegraphics[width=0.45\textwidth]{figures/1lep/TrackSyst/SystQGSRResTight_ttbar_RNN.pdf}}
%%%        \subfigure[Resolved SR W]{\includegraphics[width=0.45\textwidth]{figures/1lep/TrackSyst/SystQGSRResTight_W_RNN.pdf}}
%%%        \caption{Track multiplicity related uncertainties for \ttbar and \Wjets events in the signal regions. }
%%%    \label{fig:1lep_TrackUncCR}
%%%\end{figure}
%%%
%%%\begin{figure}[ht]
%%%    \centering
%%%        \subfigure[Merged HP SR Signal]{\includegraphics[width=0.45\textwidth]{figures/1lep/TrackSyst/SystQGSRHP_Signal_RNN.pdf}}
%%%        \subfigure[Merged LP SR Signal]{\includegraphics[width=0.45\textwidth]{figures/1lep/TrackSyst/SystQGSRLP_Signal_RNN.pdf}}
%%%        \subfigure[Resolved SR Signal]{\includegraphics[width=0.45\textwidth]{figures/1lep/TrackSyst/SystQGSRResTight_Signal_RNN.pdf}}
%%%        \caption{Track multiplicity related uncertainties for signal events in the signal regions. }
%%%    \label{fig:1lep_TrackUncSR}
%%%\end{figure}


%both from quick statistics test 
%both from past studies in the VV semi-leptonic resonant search \cite{Bachas:2646593}.


