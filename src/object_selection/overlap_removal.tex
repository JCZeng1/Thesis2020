%\subsection{Overlap Removal}

We utilize the standard overlap removal tools, AssociationUtils~\cite{Farrell2023AssociationUtils}, to manage the overlaps between analysis-level objects (electrons, muons, jets, etc.). This approach ensures that the same energy deposits are not used in the reconstruction of multiple analysis-level objects. The specific overlap removal tools employed in this analysis are summarized in Table~\ref{tab:OR}.
The overlap between small-R and large-R jets will be addressed in the later stages of the selection process by adopting a ``Merged over Resolved'' regime categorization, as detailed in Sections \ref{subsec:sr_selection} and \ref{subsubsec:merged_jets_selection}.

\begin{table}[ht]
\caption{The sequential application of overlap removal conditions. $\Delta R$ is calculated using rapidity by default.}
\label{tab:OR}
\begin{center}
\small
%\resizebox{\textwidth}{!}{
\begin{tabular}{|c|c|c|}
\hline
 Reject & Against & Criteria \\\hline
 electron & electron & shared track, $p_{T,1} < p_{T,2}$ \\
 muon     & electron & is calo-muon and shared ID track \\
 electron & muon     & shared ID track \\
 jet      & electron & $\Delta R <$ 0.2 \\ %Not a bjet and $\Delta R <$ 0.2] 
 electron & jet      & $\Delta R <$ 0.4 \\
 jet      & muon     & NumTrack $<$ 3 and (ghost-associated or $\Delta R <$ 0.2) \\
 muon     & jet      & $\Delta R <$ min(0.4, 0.04 + $10\,\text{GeV}/p_{T,\mu}$) \\ %$\Delta R <$ 0.4
 large-R-jet  & electron & $\Delta R <$ 1.0 \\
\hline
\end{tabular}
%}
\end{center}
\end{table}


