%\subsection{Muon selection}
\label{subsec:muon_selection}
Muons are chosen similarly to electrons, with the key distinction being the use of the muon spectrometer instead of the calorimeter. They must meet comparable momentum and isolation criteria, plus a consistency check aligning tracks from the inner detector to the muon spectrometer~\cite{Aad2023}. Muons failing basic quality standards during data collection are excluded.
The criteria for muon identification and isolation are optimized to reduce the presence of non-prompt muons.
This analysis uses two muon definitions, Tight and Loose, summarized in Table~\ref{tab:muon_selection}.
\begin{itemize}
\item ``Tight'' muon: used to select $W\to \mu\nu$ candidate.
\item ``Loose'' muon: used to veto events with additional leptons.
\end{itemize}

\begin{table}[ht]
\caption{Summary of Muon Selections}
\label{tab:muon_selection}
\begin{center}
%\resizebox{\textwidth}{!}{
\begin{tabular}[ht]{|c|c|c|}
\hline
  \emph{Criteria} & \emph{Loose} & \emph{Tight}\\
\hline
  Pseudorapidity range & \multicolumn{2}{c|}{$|\eta|<2.5$} \\
\hline
Momentum Calibration & \multicolumn{2}{c|}{Sagitta Correction~\cite{Aad2023} Used} \\
\hline
  Transverse momentum & $\pt > 7~\GeV$ & $\pt > 28~\GeV$ \\
\hline
  $d_0$ Significance Cut & \multicolumn{2}{c|}{$|d_{0}^{BL}(\sigma)|<3$} \\
\hline
  $z_0$ Cut & \multicolumn{2}{c|}{$|z_{0}^{BL} \sin\theta| < 0.5$~mm} \\
\hline
  Selection Working Point & Loose & Medium\\
\hline
  Isolation Working Point & FixedCutLoose at $\pt<100\,\GeV$                   & FixedCutTightTrackOnly \\
                          & and no isolation requirement at $>100\,\GeV$ & \\
\hline
\end{tabular}
%}
\end{center}
\end{table}

