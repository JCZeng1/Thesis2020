\clearpage
\subsection{VBS Selection}
\label{subsec:vbs_selection}

In all three channels events are required to contain a small-R jet pair as a candidate VBS jets, we refer to them as Tag Jets. 

The hadronically-decaying $W/Z$ candidate is reconstructed as either two small-$R$ jets~($j$)
or one large-$R$ jet~($J$), as explained in the following sections \ref{subsubsec:merged_jets_selection}-\ref{subsubsec:resolved_jets_selection}; therefore, each event should contain at least four small-$R$ jets in the resolved analysis or one large-$R$ jet and
two small-$R$ jets in the merged analysis. 

The strategy to select first tag jets and then signal jets which came from $W/Z$ from the remaining jets is chosen; 
this is motivated from the preference to focus on the production topology first (VBS jets) 
and then move to identify the objects related to the decay part of the target process. 
%Furthermore, this choice is consistent what the strategy performed in the VV semi-leptonic resonant search \cite{Bachas:2646593}.

The tagging jets algorithm is quite usual to what is done in analyses in ATLAS.

Tagging jets are required, 
firstly, 
%to be non-$b$-tagged; this would help in order to suppress the contribution of diagrams with a $Wtb$ vertex
%(especially the electroweak $t\bar{t}$ production) in the electroweak $VVjj$ production. 
%Furthermore, 
%fJVT selection has been applied to the small-R jets; 
to pass the fJVT selection, the Loose fJVT WP is used.
%the jets are required to pass the Loose fJVT WP as additional criteria to be selected as tagging jets. 
The Loose WP has been chosen for simplicity, indeed, it is the default WP recommended 
and the related SF are by default available in our analysis framework; 
furthermore, no significant difference with respect to the Tight WP has been observed, 
more details are given in dedicated studies reported in appendix \ref{app:fjvt}.
Furthermore, b-veto has been investigated to be applied on all the jets of the collection
before pairing the candidates, since, no significant impacts has been found,
as documented in the Appendix \ref{app:tagjet_bveto}, no b-jet requirement is applied on the taggging jets.

The tagging jets candidates must be in the opposite hemispheres, 
$\eta_{\mathrm{tag}\ j_1} \cdot \eta_{\mathrm{tag}\ j_2} < 0$,
and to have the highest dijet invariant mass among all the possible pairs of small-R jets in the event, 
that have passed already 
%b-tagging veto and 
the fJVT requirement, as mentioned above.
 
After the tagging jet pair are selected, it is required that both tagging jets should have \pT$>$30~\GeV and that the invariant mass of the two tagging jets system is greater than 400~\GeV; 
%we rely on the optimisation studies done in the previous round of the analysis on the choice of both \pT and invariant mass cuts. 
As an example, the \mjjtag distribution for both data and MC samples is shown before the cut in the \zlep channel SRs Figure \ref{fig:0lepMjj}; 
the \mjjtag re-weighting is already applied and it will be described in section \ref{subsec:mjj_reweight}.

\begin{figure}[ht]
    \centering
    \subfigure[merged HighPurity]{\includegraphics[width=0.3\textwidth]{figures/0lep/cutflow/nominal/merged/plots/soverb_individual_SRVBS_HP_MTagMerJets400_MTagMerJets_cutflow.pdf}}
    \subfigure[merged LowPurity]{\includegraphics[width=0.3\textwidth]{figures/0lep/cutflow/nominal/merged/plots/soverb_individual_SRVBS_LP_MTagMerJets400_MTagMerJets_cutflow.pdf}}
    \subfigure[resolved]{\includegraphics[width=0.3\textwidth]{figures/0lep/cutflow/nominal/merged/plots/soverb_individual_SRVBS_Res_MTagResJets400_MTagResJets_cutflow.pdf}}
    \caption{0-lepton $m(jj)^\text{tag}$ selection. The distributions are shown for the three SR selections; the specific definition of each of them (HighPurity, LowPurity, Resolved) will be given in the following Section \ref{subsec:sr_selection}. Entering are events passing the respective selection up to the point of the $m(jj)^\text{tag}$ cut which is shown by the vertical line.} 
    \label{fig:0lepMjj}
\end{figure}



