%%\clearpage
%%\subsection{VBS Selection}
\label{subsec:vbs_selection}

In the VBS selection, all events are required to have a pair of small-$R$ jets, designated as Tag Jets, which serve as candidates for VBS jets.
The hadronically-decaying $W/Z$ candidate is reconstructed using either one large-$R$ jet~($J$) or two small-$R$ jets~($j$), as detailed in Sections~\ref{subsubsec:merged_jets_selection} and~\ref{subsubsec:resolved_jets_selection}. 
In the merged case, each event must include one large-$R$ jet and two small-$R$ jets, while in the resolved case, the requirement is for at least four small-$R$ jets. 
Therefore, the resolved case involves a more complex selection strategy. Tag Jets are selected first, which are essential for defining the VBS jet candidates. Next, among the remaining jets, those emanating from the $W/Z$ decay are identified as signal jets. This approach prioritizes establishing the forward topology (VBS jets) before identifying the decay products of the target process.

The Tag Jets algorithm employed in this analysis adheres to standard ATLAS practices. As discussed in Section~\ref{subsubsec:resolved_jets_selection}, key requirements for Tag Jets include passing the fJVT selection, specifically using the Loose fJVT working point (WP). This Loose WP is selected for its simplicity and because it is the recommended default WP. Additionally, the corresponding scale factors (SFs) are readily available in our analysis framework.

Tag Jets candidates are required to be in opposite hemispheres, satisfying $\eta_{\mathrm{tag}\, j_1} \cdot \eta_{\mathrm{tag}\, j_2} < 0$. They should also have the highest dijet invariant mass among all pairs of small-$R$ jets in the event that meet the fJVT requirements, as previously described. This criterion ensures the selection of the most physically significant jet pairs for analysis.

Once the Tag Jet pair is selected, the following criteria are applied to both jets to enhance the signal-to-background ratio:
\begin{itemize}
    \item Each Tag Jet must have a transverse momentum (\pt) greater than \SI{30}{\GeV}.
    \item The invariant mass of the two Tag Jet system should exceed \SI{400}{\GeV}.
\end{itemize}

This approach is based on optimization studies conducted in the previous round of the analysis, which utilized the 36 \,\ifb dataset~\cite{Ryzhov:2310214}, to inform our choice of both \pT and invariant mass cuts.

