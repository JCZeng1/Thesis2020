\onecolumn
%-------------------------------------------------------------------------------
\section{Changes}
\label{sec:change}
%-------------------------------------------------------------------------------

\textbf{Version 01-08-01} of \Package{atlaslatex} includes quite a few definitions from the Jet/ETmiss group.
A new style file has been created \File{atlasjetetmiss.sty} that is not included by default.
Some of the definitions from the Jet/ETmiss group are of more general use and so have been merged into existing style files:
\begin{description}
\item[atlasmisc.sty] List of Monte Carlo generators expanded:
  \Macro{POWHEGBOX}, \Macro{POWPYTHIA}.
  Add MC macros with suffix \enquote{V} for version number.
  \Macro{kt}, \Macro{antikt}, \Macro{Antikt}, \Macro{LO}, \Macro{NLO}, \Macro{NLL}, \Macro{NNLO},
  \Macro{muF}, \Macro{muR}.
  Added macros \Macro{Runone}, \Macro{Runtwo}, \Macro{Runthr},
  Added \Macro{radlength} and \Macro{StoB}.
  Added some standard $b$-tagging terms:
  \Macro{btag}, \Macro{btagged}, \Macro{bquark}, \Macro{bquarks}, \Macro{bjet}, \Macro{bjets}.
\item[atlasparticle.sty] Now includes \Macro{pp}, \Macro{enu}, \Macro{munu},
\item[atlasprocess.sty] Added \Macro{Zbb}, \Macro{Ztt},
  \Macro{Zlplm}, \Macro{Zepem}, \Macro{Zmpmm}, \Macro{Ztptm},
  \Macro{tWb}, \Macro{Wqqbar},
  \Macro{Wlnu}, \Macro{Wenu}, \Macro{Wmunu},
  \Macro{Wjets}, \Macro{Zjets}.
  The definition of \Macro{Hllll} was corrected.
\item[atlasheavyion.sty] \Macro{pp} moved to \File{atlasparticle.sty}.
\end{description}

This version also introduced the (optional) use of the \Package{heppennames} package.
The style files \File{atlashepparticle.sty} and \File{atlashepprocess.sty}
are intended to replace \File{atlasparticle.sty} and \File{atlasprocess.sty}.
Several particle definitions were removed from the \Package{atlasparticle} package,
as they just enable a few Greek letters: $\pi$, $\eta$ and $\psi$ to be used directly in text mode.
In addition, the primed $\Upsilon$ resonances, e.g.\ $\Upsilon''$,
as well as $D^{**}$ were removed.
as the official names are \Ups{3} etc.,

The definitions of \MeV, \GeV\ etc.\ in \texttt{atlasunit.sty} were updated in order to remove the \texttt{if} tests in them.
The \texttt{if} tests caused a problem in a paper draft, although the reason was not understood.
The new definitions do not introduce extra space before the unit in math mode.


%-------------------------------------------------------------------------------
\section{Old macros}
\label{sec:old}
%-------------------------------------------------------------------------------

With the introduction of \Package{atlaslatex} several macro names have been changed to make them more consistent.
A few have been removed. The changes include:
\begin{itemize}
\item Kaons now have a capital \enquote{K} in the macro name, e.g.\ \verb|\Kplus| for \Kplus;
\item \verb|\Ztau|, \verb|\Wtau|, \verb|\Htau| \verb|\Atau| have been replaced by
  \verb|\Ztautau|, \verb|\Wtautau|, \verb|\Htautau| \verb|\Atautau|;
\item \verb|\Ups| replaces \verb|\ups|;
  the use of \verb|\ups| to produce $\Upsilon$ in text mode has been removed;
\item \verb|\cm| has been removed, as it was the only length unit defined for text and math mode;
\item \verb|\mass| has been removed, as \verb|\twomass| can do the same thing and the name is more intuitive;
\item \verb|\mA| has been removed as it conflicts with \Package{siunitx} Version 1, which uses the name
  for milliamp.
\item \Macro{mathcal} rather than \Macro{mathscr} is recommended for luminosity and aplanarity.
\end{itemize}

Quite a few macros are more related to \Zboson physics than they are to LHC physics and have
been moved to the \File{atlasother.sty} file, which is not included by default.
There are also macros for various decay processes, \File{atlasprocess.sty} which are not included by default,
but may be useful for how you can define your favourite process.

It used to be the case that you had to use \verb|\MET{}| rather than just \verb|\MET| to get the spacing right,
as somehow \Package{xspace} did not do a good job for \met.
However, with the latest version of the packages both forms work fine.
You can compare \MET and \MET\ and see that the spacing is correct in both cases.
